% Seleciona o idioma do documento (conforme pacotes do babel)
%\selectlanguage{english}
\selectlanguage{brazil}

% Retira espaço extra obsoleto entre as frases.
\frenchspacing 

\newpage

% ==============================================
% ELEMENTOS PRÉ-TEXTUAIS
% ==============================================
\pretextual

% ----------------------------------------------
% Capa
% ----------------------------------------------
%\imprimircapa
% Capa personalizada sem o uso de \imprimircapa
\begin{capa} 
    \begin{center}
        \begin{minipage}{1\textwidth} 
            \large\centering\makebox[\textwidth]{\includegraphics[scale=1]{imagens/Logo UNIP.png}}
        \end{minipage}
    \end{center}
    \begin{center}
        \LARGE\textbf{UNIP -- UNIVERSIDADE PAULISTA\\} 
        \LARGE Curso de Ciência da Computação\\
        \vfill
        \ABNTEXchapterfont\Large\textbf{\MakeUppercase{ATIVIDADES PRÁTICAS SUPERVISIONADAS - APS}}
        \\\small{AS TÉCNICAS CRIPTOGRÁFICAS, CONCEITOS, USOS E APLICAÇÕES\\UTILIZANDO A TÉCNICA TRIPLE DES} 
        \vfill
        \normalsize
        Daniel Chrispim - F263614\\
        Daniel Lima - N621184\\
        Gabriel De Paula Seki - N6853H6\\
        Maisa Serpa Castro Moreira - F17EFC7\\
        Matheus Lopes - N620ED0\\
        Vinícius Amorim - N641JC
        \vfill
        São José dos Campos, \today 
    \end{center}
\end{capa}

% ----------------------------------------------
% Folha de rosto
% ----------------------------------------------
% folha de rosto personalizada sem uso de \imprimirfolhaderosto
\makeatletter
\renewcommand{\folhaderostocontent}{
\begin{center}
    \begin{center}
        \begin{minipage}{1\textwidth} 
            \large\centering\makebox[\textwidth]{\includegraphics[scale=1]{imagens/Logo UNIP.png}}
        \end{minipage}
    \end{center}
    
  \vspace*{\fill}%\vspace*{\fill}
  \begin{center}
  \ABNTEXchapterfont\Large\textbf{\MakeUppercase{ATIVIDADES PRÁTICAS SUPERVISIONADAS - APS}}
  \\\small{AS TÉCNICAS CRIPTOGRÁFICAS, CONCEITOS, USOS E APLICAÇÕES\\UTILIZANDO A TÉCNICA TRIPLE DES}
  \end{center}
  \vspace*{\fill}
  
    \hspace{.45\textwidth}
    \begin{minipage}{.5\textwidth}
    \SingleSpacing
    {Atividades Práticas Supervisionadas do 2\degree\ Semestre do Curso de Ciência da Computação da \textbf{Universidade Paulista UNIP.}}
    \end{minipage}%
    \vspace*{\fill}


  \hspace{.45\textwidth}
  \begin{minipage}{.5\textwidth}
	{\textbf{Coordenador:} Prof. Fernando A. Gotti}%
  \end{minipage}%
 
  
  \hspace{.45\textwidth}
  \begin{minipage}{.5\textwidth}
	{\textbf{Prof. Responsável:} Luiz Gustavo Miranda Pinto}%
  \end{minipage}%

  
  \vspace*{\fill}
  %{\abntex@ifnotempty{\imprimirinstituicao}{\imprimirinstituicao\vspace*{\fill}}}

  São José dos Campos, \today
\end{center}
}
\makeatother

% Folha de rosto (o * indica que haverá a ficha bibliográfica)

\imprimirfolhaderosto

% ||||||||||||||||||||||||||||||||||||||||||||||
% RESUMOS
% ||||||||||||||||||||||||||||||||||||||||||||||

% ----------------------------------------------
% Resumo em português
% ----------------------------------------------
% Importante: De acordo com a NBR6024 as palavras-chaves devem ser separadas entre si por ponto e devem ter somente a primeira palavra escrita com letra maiúscula
% \setlength{\absparsep}{18pt} % ajusta o espaçamento dos parágrafos do resumo
% \begin{resumo}
    
% 	\vspace{\onelineskip}
 
% 	\noindent 
% 	\textbf{Palavras-chaves}: Sistemas microeletromecânicos (MEMS). Isoladores e isolamentos elétricos. Fontes Chaveadas. Fonte chaveada isolada. Isolação elétrica. Fonte de alimentação de baixo custo.
% \end{resumo}

% ----------------------------------------------
% Resumo em inglês
% ----------------------------------------------
% Importante: De acordo com a NBR6024 as palavras-chaves devem ser separadas entre si por ponto e devem ter somente a primeira palavra escrita com letra maiúscula
% \begin{resumo}[Abstract]
% \begin{otherlanguage*}{english}
% 	\lipsum[8]
    
% 	\vspace{\onelineskip}
 
% 	\noindent 
% 	\textbf{Key-words}: Microelectromechanical systems (MEMS). Electronic insulation. Switching power supply. Isolated power supply. Electric isolated circuitry. Low cost power supply.
% \end{otherlanguage*}
% \end{resumo}

% ----------------------------------------------
% inserir lista de ilustrações
% ----------------------------------------------
% \pdfbookmark[0]{\listfigurename}{lof}
% \listoffigures*
% \cleardoublepage

% Diferentes tipos de listas podem ser criadas por meio de macros do memoir.

% ----------------------------------------------
% inserir lista de tabelas
% ----------------------------------------------
% \pdfbookmark[0]{\listtablename}{lot}
% \listoftables*
% \cleardoublepage

% ----------------------------------------------
% inserir lista de abreviaturas e siglas
% ----------------------------------------------
% Importante: As abreviaturas e siglas devem estar em ordem alfabética
% \begin{siglas}
%   \item[ABNT] Associação Brasileira de Normas Técnicas
%   \item[abnTeX] ABsurdas Normas para TeX
% \end{siglas}

% ----------------------------------------------
% inserir lista de símbolos
% ----------------------------------------------
% Importante: Os símbolos devem estar na ordem de aparecimento no texto.
% \begin{simbolos}
%   \item[$ \Gamma $] Letra grega Gama
%   \item[$ \Lambda $] Lambda
%   \item[$ \zeta $] Letra grega minúscula zeta
%   \item[$ \in $] Pertence
% \end{simbolos}

% ----------------------------------------------
% inserir o sumário
% ----------------------------------------------
\pdfbookmark[0]{\contentsname}{toc}
\tableofcontents*
\cleardoublepage
