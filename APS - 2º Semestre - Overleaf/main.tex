\documentclass[
	% -- opções da classe memoir --
	12pt,		% tamanho da fonte
	%openright,	% capítulos começam em pág ímpar (insere página vazia caso preciso)
	oneside,	% para impressão em verso e anverso. Oposto a oneside
	a4paper,	% tamanho do papel. 
	% -- opções da classe abntex2 --
	chapter=TITLE,		% títulos de capítulos convertidos em letras maiúsculas
	%section=TITLE,		% títulos de seções convertidos em letras maiúsculas
	%subsection=TITLE,	% títulos de subseções convertidos em letras maiúsculas
	%subsubsection=TITLE,% títulos de subsubseções convertidos em letras maiúsculas
	% -- opções do pacote babel --
	english,	% idioma adicional para hifenização
	brazil		% o último idioma é o principal do documento
	]{abntex2}
%----------------------------------------------------------------------------------------
% Definição dos Packages
%----------------------------------------------------------------------------------------	
%\usepackage[utf8]{inputenc}	                    % Codificacao do documento (conversão automática dos acentos)
\usepackage[T1]{fontenc}	                    % Selecao de codigos de fonte. Afeta separação de sílabas
\usepackage[brazilian,hyperpageref]{backref}    % Paginas com as citações na bibl
\usepackage[alf]{abntex2cite}                   % Citações padrão ABNT
\usepackage{helvet}                             % Usado para definir a fonte em Helvetica
\usepackage{lastpage}		                    % Usado pela ficha catalografica
\usepackage{indentfirst}	                    % Indenta o primeiro paragrafo de cada seção
\usepackage{microtype} 		                    % Para melhorias de justificação
\usepackage{bibentry}   	                    % para inserir refs. bib. no meio do texto
\usepackage{listings}                           % Define as listas (numeradas)
\usepackage{tikzsymbols}                        % Simbolos matemáticos (usado pra colocar o logo)
\usepackage{gensymb}                            % Pra poder usar o º (\degree)
%----------------------------------------------------------------------------------------
% Execução dos Packages
%----------------------------------------------------------------------------------------
\renewcommand{\familydefault}{\sfdefault}       % Define em Helvetica
\makeindex                                      % compila o indice

\lstset{
    numbers=left, 
    stepnumber=1, 
    firstnumber=1, 
    numberstyle=\tiny, 
    extendedchars=true,
    breaklines=true, 
    frame=tb, 
    basicstyle=\footnotesize, 
    stringstyle=\ttfamily, 
    showstringspaces=false
}

% Configurações do pacote backref 
\renewcommand{\backrefpagesname}{Citado na(s) página(s):~}
\renewcommand{\backref}{}                       % Texto padrão antes do número das páginas
\renewcommand*{\backrefalt}[4]{                 % Define os textos da citação
	\ifcase #1
		
	\or
		Citado na página #2.
	\else
		Citado #1 vezes nas páginas #2.
	\fi}
	
% Espaçamentos entre linhas e parágrafos 
\linespread{1.5}                                % Espaçamento da linha
\setlength{\parindent}{1.3cm}                   % O tamanho do parágrafo
\setlength{\parskip}{0.2cm}                     % Controle do espaçamento entre um parágrafo e outro

\begin{document}

\input{1-pretextual}
%----------------------------------------------------------------------------------------
% Introdução, Objetivos e Justificativa
%----------------------------------------------------------------------------------------
\textual\newpage
\chapter{\textbf{Introdução}}

\section{\textbf{Objetivos Gerais}}
\par A criptografia é considerada uma técnica científica de mensagem cifrada ou em código, sendo um dos principais mecanismos de segurança que pode utilizar para se proteger dos riscos associados à utilização de internet. O 3DES foi escolhido por ser uma das cifras de blocos mais utilizadas no mundo, além de ter maior segurança de dados evitando o fácil acesso por força bruta (tentativa e erro).

%----------------------------------------------------------------------------------------
\section{\textbf{Objetivos Especificos}}
\par O foco principal do trabalho é desenvolver um programa de login com criptografia \textbf{3DES} utilizando a linguagem \textbf{PYTHON}, muito utilizado no cenário atual, grandes empresas trocam informações sigilosas diariamente e devido a isso é extremamente importante manter seus dados seguros. 

\par O \textbf{Triple Data Encryption (3DES)} utilizado em nosso trabalho, foi desenvolvido pela International Business Machines (IBM) em 1974 e adotado como padrão em 1977, é um padrão de criptografia de chave simétrica baseado em uma rede Feistel, utiliza 3 chaves de 64bits, embora cada chave use apenas 56 bits, enquanto os outros 8 bits são usados para verificar a paridade.

\par O tamanho máximo efetivo da chave é 168 bits.Os dados são criptografados com a primeira chave, descriptografados com a segunda chave e finalmente criptografados com a terceira chave novamente, fornecendo maior segurança. 

%----------------------------------------------------------------------------------------
\newpage
\section{\textbf{Justificativa}}
\par A criptografia é considerada uma técnica científica de mensagem cifrada ou em código, sendo um dos principais mecanismos de segurança que pode utilizar para se proteger dos riscos associados à utilização de internet. O 3DES foi escolhido por ser uma das cifras de blocos mais utilizadas no mundo, além de ter maior segurança de dados evitando o fácil acesso por força bruta (tentativa e erro).

%----------------------------------------------------------------------------------------
% Tema Escolhido
%----------------------------------------------------------------------------------------
\newpage
\chapter{\textbf{Tema Escolhido}}

\section{\textbf{O que é criptografia?}}
\par A criptografia é uma prática utilizada para proteger dados de quaisquer tipos de invasões, ela faz a utilização de codificações onde um algoritmo é responsável em mudar os dados, para que eles não possam ser lidos e somente com uma chave específica esses dados voltam ao seu formato original.

\par Hoje em dia com o avanço da tecnologia a criptografia é uma parte muito importante para a segurança, pois cada vez mais temos malwares que podem “roubar” desde suas senhas de redes sociais e até informações sigilosas do governo.
Como já citado anteriormente, para que seja possível decifrar a criptografia utilizamos uma “key” e cada tipo de criptografia utiliza uma certa quantidades de bits na geração de sua chave, quanto mais bits mais complexo é sua chave, por exemplo, se temos uma chave com 128 bits significa que temos 2128 chaves possíveis para para decifrá-los, quanto maior a quantidade de bits mais seguro será a codificação.

%----------------------------------------------------------------------------------------
\newpage
\section{\textbf{Tipos de criptografia}}
\par A criptografia tem uma ampla variedade de tipos e cada uma tem seu próprio diferencial, tanto como a aplicação ou como sua dificuldade de utilização. Aqui iremos citar alguns dos tipos mais conhecidos e utilizado atualmente.
 
\textbf{Data Encryption Standard (DES):} É um tipo de criptografia mais vulnerável pois se baseia em fazer 16 ciclos de codificações para proteger seus dados e essa proteção utiliza aproximadamente 56 bits.
Essa criptografia é uma das primeiras a serem utilizadas, sendo assim a mais conhecida e utilizada no mundo, porém sua fraqueza é a fácil decodificação com programas que utilizam uma técnica de força bruta, onde esse programa testa diversas chaves automaticamente durante horas até conseguir a correta. 

\textbf{Advanced Encryption Standard (AES):} Este algoritmo é bastante conhecido pois várias instituições utilizarem e até mesmo o governo do Estados Unidos usa ele como padrão.
A proteção dele pode variar de acordo com a segurança que você precisa, ele irá te oferecer proteções desde 128 ou até mesmo 192 e 256 bits, assim o tornando um algoritmo extremamente eficaz e confiável.

\textbf{Rivest-Shamir-Adleman (RSA):} Conhecido por sua extrema segurança, ele trabalha de uma forma diferente dos outros tipos de algoritmos de criptografia, sua complexidade é dada por fazer a utilização de duas chaves, uma pública e uma privada.
A chave pública tem como função cifrar as mensagens e somente com a chave privada você irá conseguir decifrar as mensagens. Essa tecnologia está bastante presente no nosso dia a dia, desde quando enviamos um e-mail para alguém até o momento que fazemos uma compra online.

\par Existem diversos tipos de criptografias, alguns já foram citados mas o mercado tem uma ampla variedade de algoritmos tais como: DESX, Camellia, Blowfish e etc, cada um pode atender o seu cliente de acordo com sua necessidade.

%----------------------------------------------------------------------------------------
\newpage
\subsection{\textbf{Criptografia simétrica ou chave privada}}

\par A criptografia simétrica ou chave privada ele trabalha na utilização de uma única chave onde essa chave é compartilhada entre o emissor e o destinatário que as guarda de forma sigilosa. Essa chave vai definir como o algoritmo vai cifrar o conteúdo utilizando uma cadeia própria de bits.

\par A vantagem de fazer a utilização da mesma é manter uma comunicação de várias pessoas e uma boa performance da criptografia e caso essa chave seja perdida ou comprometida, basta efetuar uma nova chave assim não precisando alterar o código inicial.

\par Sua segurança vai de acordo com o tamanho da chave utilizada, mas nem tudo é um mar de rosas, sua segurança é falha pois para grandes números de usuários utiliza se duas chaves, ou seja para 10 usuários vão ser geradas 20 chaves, assim fazendo que a gestão de chaves seja complexa.

%----------------------------------------------------------------------------------------
\subsection{\textbf{Criptografia assimétrica ou chave pública}}

\par Diferente da chave privada a chave pública ou chave assimétrica trabalha com dois tipos de chave de segurança, essas chaves são usada uma para cifrar (chave pública) e uma chave para decifrar (chave privada), assim facilita e melhora a segurança, por exemplo se um destinatário envia um conteúdo criptografado, ele só irá utilizar a chave pública, agora o destinatário só irá precisar de usar a chave privada para poder descriptografar o conteúdo.

\par Esse sistema garante uma melhor privacidade do usuário, assim garante uma grande confiabilidade na utilização. Um dos algoritmos que utiliza esse tipo de chave é o RSA que se baseia em uma multiplicação de números primos para gerar sua chave pública.

%----------------------------------------------------------------------------------------
\newpage
\subsection{\textbf{Certificado digital}}

\par Baseia-se numa codificação criptográfica onde um software faz operações matemática para gerar uma chave secreta para cada mensagem. Assim o emissor manda um conteúdo cifrado onde todo o conteúdo estará misturado ou embaralhado e quando o destinatário receber ele terá que desembaralhar o conteúdo, isso só pode ocorrer se o destinatário tiver posse da chave correta para decodificação do código.

%----------------------------------------------------------------------------------------
\subsection{\textbf{Assinatura digital}}

\par É uma codificação que trabalha com funções matemáticas, assim fazendo uma assinatura digital, onde o receptor tem a certeza que o conteúdo recebido não teve alterações na transmissão por terceiros, para que isso funcione o cálculo não pode somente usar cálculos matemáticas, mas deve complementar com informações que somente o remetente saiba, então o receptor ao receber tem que conferir se as informações estão iguais, nesse caso são geradas duas chaves, uma privada e outra pública. Um exemplo da utilização é o imposto de renda.

%----------------------------------------------------------------------------------------
\newpage
\subsection{\textbf{Funções hashing}}

\par Funções hashing ou somente Hash é um tipo de algoritmo que faz o mapeamento de entradas de dados com comprimento variável para um comprimento fixo, a função tem como principal função a garantia de que o conteúdo enviado não tenha sido modificado em seu envio.

\par Ela irá trabalhar com uma entrada de “n” itens, essa entrada normalmente é variável ou até o tamanho máximo da função hash, ela é muito utilizada pois são eficientes na verificação de dados, a função sempre irá retornar seu valor hash para qualquer mensagem.

\par A função na prática funciona da seguinte forma: começa com dois blocos de tamanhos fixos de dados de hash ( o tamanho do bloco vai variar de acordo com o algoritmo, sendo utilizado 128 a 512 bits), então a função irá pegar as informações e colocar em um ciclo, este ciclo irá rodar até que toda mensagem seja modificada, então depois disso ela vai para segunda entrada da função hash, o resultado desse ciclo e vai para um terceiro ciclo, assim sucessivamente, quando terminar todos os ciclos teremos o valor Hash que é totalmente diferente do que se iniciou.

%----------------------------------------------------------------------------------------
\newpage
\section{\textbf{Triple DES (3DES)}}

\par Com a facilidade em burlar a criptografia DES houve a necessidade da criação de outro tipo de proteção que seja mais forte contra os hackers, assim foi criado o Triple DES.
\par A sua proteção é baseada em em utilizar três chaves que contém 56 bits, por isso seu nome, ela é três vezes mais protegida que o DES, assim tendo uma proteção de 168 bits.

%----------------------------------------------------------------------------------------
\subsection{\textbf{Histórico do 3DES}}

\par O intuito da criação desse tipo de criptografia foi para aprimoração da segurança de sua utilização, ela é uma versão mais segura do Data Encryption Standard e foi criada no ano de 1974 pela empresa International Business Machines Corporation (IBM), uma empresa do ramo de informática e tecnologia, porém sua utilização se tornou padrão somente três anos depois.

%----------------------------------------------------------------------------------------
\subsection{\textbf{Descrição da técnica}}

\par A técnica utilizada pelo 3DES se baseia na utilização de tres chaves criptográficas, a primeira chave faz a criptografia da mensagem, a segunda faz a decifração da mensagem criptografada e a última torna a criptografia o resultado da operação.

%----------------------------------------------------------------------------------------
\newpage
\section{\textbf{Software Construído}}

\par O software é responsável por controlar a entrada de pessoas em uma determinada área privada, cada pessoa autorizada terá um usuário e senha que serão criados e inseridos dentro do Banco de Dados, essas senhas serão criptografadas utilizando a técnica de criptografia 3DES. No algoritmo temos uma chave "Sixteen byte key" que possui 16 bits.

\par\textbf {Software de criação de usuário:} Quando o usuário finalizar seu cadastro de usuário e senha, a senha será encriptados usando a key e o IV, logo em seguida serão convertidos em formato binário(senha e IV, usuário mantém formato texto) e enviados para o Banco de Dados.

\par\textbf {Software de acesso:} Serão inseridos usuário e senha, com isso o algoritmo irá buscar no banco de dados e verificar se o usuário está correto, se estiver correto, o algoritmo vai no banco de dados e testa a senha inserida com todos os IV's guardados no banco, converter de binário para string, até achar o que ao decriptar seja igual ao inserido, caso usuário e senha inseridos sejam iguais aos guardados no Banco de Dados, será liberado para entrar na área privada com a informação das Substâncias e Tripulantes.

%----------------------------------------------------------------------------------------
% Estrutura do Software
%----------------------------------------------------------------------------------------
\newpage
\chapter{\textbf{Estrutura do Software}}

\par O software é composto de um menu de login e senha, e um menu de acesso a dados. O layout escolhido pro software é a GUI simples do Tkinter, que é uma camada fina orientada a objeto sobre Tcl/Tk feita para interfaces gráficas. Os módulos do Tkinter foram importados no Pycharm e no VScode por alguns integrantes da APS, mas todos utilizando a linguagem Python para importar os módulos. Foi instalada também a biblioteca do Pycryptodome a fim de  importar os módulos de criptografia 3DES e implementar o algoritmo de codificação. Para implementar a criptografia ao banco de dados, utilizamos a linguagem estruturada de consulta SQLite, e também o editor SQLite Studio.

%----------------------------------------------------------------------------------------
% Código Fonte
%----------------------------------------------------------------------------------------
\newpage
\chapter{\textbf{Código Fonte}}
\renewcommand\lstlistingname{\textbf{Arquivo}}

\lstinputlisting[language=python, label=acessar, caption={BD\_acesso.py}]{codigos/BD_acesso.py}
\newpage
\lstinputlisting[language=python, label=acessarGUI, caption={GUI\_acesso.py}]{codigos/GUI_acesso.py}
\newpage
\lstinputlisting[language=python, label=criar, caption={BD\_criar.py}]{codigos/BD_criar.py}
\newpage
\lstinputlisting[language=python, label=criarGUI, caption={GUI\_criar.py}]{codigos/GUI_criar.py}

%----------------------------------------------------------------------------------------
% Código Fonte
%----------------------------------------------------------------------------------------
\newpage
\chapter{\textbf{GLOSSARIO}}

\par \textbf{Banco de dados:} Bancos de dados ou bases de dados são conjuntos de arquivos relacionados entre si com registros sobre pessoas, lugares ou coisas. São coleções organizadas de dados que se relacionam de forma a criar algum sentido e dar mais eficiência durante uma pesquisa ou estudo científico.

\par \textbf{Encriptar:} Em criptografia, encriptação, ou cifragem, é o processo de transformar informação usando um algoritmo de modo a impossibilitar a sua leitura a todos excepto aqueles que possuam uma identificação particular, geralmente referida como chave.

\par \textbf{International Business Machines (IBM):} A International Business Machines Corporation é uma empresa dos Estados Unidos voltada para a área de informática. A empresa é uma das poucas na área de tecnologia da informação com uma história contínua que remonta ao século XIX.

\par \textbf{Key:} Uma chave é um pedaço de informação que controla a operação de um algoritmo de criptografia. Na codificação, uma chave específica a transformação do texto puro em texto cifrado, ou vice-versa, durante a decodificação

\par \textbf{Linguagem de Programação:} A linguagem de programação é um método padronizado, formado por um conjunto de regras sintáticas e semânticas, de implementação de um código fonte - que pode ser compilado e transformado em um programa de computador, ou usado como script interpretado - que informará instruções de processamento ao computador.

\par \textbf{Malware:} Um código malicioso, programa malicioso, software nocivo, software mal-intencionado ou software malicioso, é um programa de computador destinado a infiltrar-se em um sistema de computador alheio de forma ilícita, com o intuito de causar alguns danos, alterações ou roubo de informações.

\par \textbf{Pycharm:} PyCharm é um ambiente de desenvolvimento integrado usado em programação de computadores, especificamente para a linguagem Python. É desenvolvido pela empresa tcheca JetBrains.

\par \textbf{Pycryptodome:} O PyCryptodome é uma biblioteca em pyhton que trata de implementa¸c˜oes de algoritmos de criptografia.

\par \textbf{Python:} Python é uma linguagem de programação de alto nível, interpretada, de script, imperativa, orientada a objetos, funcional, de tipagem dinâmica e forte. Foi lançada por Guido van Rossum em 1991.

\par \textbf{Feistel:} a criptografia, uma cifra de Feistel é uma estrutura simétrica usada na construção de cifras de bloco, o nome é uma homenagem ao físico e criptógrafo alemão Horst Feistel, que foi o pioneiro na pesquisa enquanto trabalhava na IBM; esta cifra é comumente conhecida como rede de Feistel.

\par \textbf{Software:} Software, é um termo técnico que foi traduzido para a língua portuguesa como logiciário ou suporte lógico, é uma sequência de instruções a serem seguidas e/ou executadas, na manipulação, redirecionamento ou modificação de um dado ou acontecimento.

\par \textbf{SQLite:} SQLite é uma biblioteca em linguagem C que implementa um banco de dados SQL embutido. Programas que usam a biblioteca SQLite podem ter acesso a banco de dados SQL sem executar um processo SGBD separado.

\par \textbf{VSCode:} O Visual Studio Code é um editor de código-fonte desenvolvido pela Microsoft para Windows, Linux e macOS. Ele inclui suporte para depuração, controle Git incorporado, realce de sintaxe, complementação inteligente de código, snippets e refatoração de código. 

\input{2-postextual}
\end{document}
