\makeglossaries
% ----------------------------------------------
% Entradas do Glossário
% ----------------------------------------------
\newglossaryentry{BD}
    {
    name={Banco de dados},
    description={Bancos de dados ou bases de dados são conjuntos de arquivos relacionados entre si com registros sobre pessoas, lugares ou coisas. São coleções organizadas de dados que se relacionam de forma a criar algum sentido e dar mais eficiência durante uma pesquisa ou estudo científico} 
    }

\newglossaryentry{Encriptar}
    {
    name={Encriptar},
    description={Em criptografia, encriptação, ou cifragem, é o processo de transformar informação usando um algoritmo de modo a impossibilitar a sua leitura a todos excepto aqueles que possuam uma identificação particular, geralmente referida como chave} 
    }

\newglossaryentry{IBM}
    {
    name={International Business Machines (IBM)},
    description={International Business Machines (IBM): A International Business Machines Corporation é uma empresa dos Estados Unidos voltada para a área de informática. A empresa é uma das poucas na área de tecnologia da informação com uma história
contínua que remonta ao século XIX} 
    }

\newglossaryentry{Key}
    {
    name={Key},
    description={Uma chave é um pedaço de informação que controla a operação de um algoritmo de criptografia. Na codificação, uma chave específica a transformação do texto puro em texto cifrado, ou vice-versa, durante a decodificação} 
    }

\newglossaryentry{LP}
    {
    name={Linguagem de Programação},
    description={A linguagem de programação é um método padronizado, formado por um conjunto de regras sintáticas e semânticas, de implementação de um código fonte - que pode ser compilado e transformado em um programa de computador, ou usado como script interpretado - que informará instruções de processamento ao computador} 
    }
    
\newglossaryentry{Malware}
    {
    name={Malware},
    description={Um código malicioso, programa malicioso, software nocivo, software mal-intencionado ou software malicioso, é um programa de computador destinado a infiltrar-se em um sistema de computador alheio de forma ilícita, com o intuito de causar alguns danos, alterações ou roubo de informações} 
    }

\newglossaryentry{Pycharm}
    {
    name={Pycharm},
    description={PyCharm é um ambiente de desenvolvimento integrado usado em programação de computadores, especificamente para a linguagem Python. É desenvolvido pela empresa tcheca JetBrains} 
    }

\newglossaryentry{Pycryptodome}
    {
    name={Pycryptodome},
    description={O PyCryptodome é uma biblioteca em python que trata de implementações de algoritmos de criptografia} 
    }

\newglossaryentry{Python}
    {
    name={Python},
    description={Python é uma linguagem de programação de alto nível, interpretada, de script, imperativa, orientada a objetos, funcional, de tipagem dinâmica e forte. Foi lançada por Guido van Rossum em 1991} 
    }

\newglossaryentry{Feistel}
    {
    name={Feistel},
    description={A criptografia, uma cifra de Feistel é uma estrutura simétrica usada na construção de cifras de bloco, o nome é uma homenagem ao físico e criptógrafo alemão Horst Feistel, que foi o pioneiro na pesquisa enquanto trabalhava na IBM; esta cifra é comumente conhecida como rede de Feistel} 
    }

\newglossaryentry{Software}
    {
    name={Software},
    description={Software, é um termo técnico que foi traduzido para a língua portuguesa como logiciário ou suporte lógico, é uma sequência de instruções a serem seguidas e/ou executadas, na manipulação, redirecionamento ou modificação de um dado ou acontecimento} 
    }

\newglossaryentry{SQLite}
    {
    name={SQLite},
    description={SQLite é uma biblioteca em linguagem C que implementa um banco de dados SQL embutido. Programas que usam a biblioteca SQLite podem ter acesso a banco de dados SQL sem executar um processo SGBD separado} 
    }

\newglossaryentry{VSCode}
    {
    name={VSCode},
    description={O Visual Studio Code é um editor de código-fonte desenvolvido pela Microsoft para Windows, Linux e macOS. Ele inclui suporte para depuração, controle Git incorporado, realce de sintaxe, complementação inteligente de código, snippets e refatoração de código} 
    }
% ----------------------------------------------
% Configurações do glossário
\renewcommand*{\glsseeformat}[3][\seename]{\textit{#1}  
\glsseelist{#2}}